\documentclass[notebook]{guildcamp2} %% [notebook] or [greennotebook]
\begin{document}

\startnotebook{\nStoneGaze{}}

\begin{page}{first}
You need to talk to people who might know how to cure you.  The Loremasters pride themselves on arcane knowledge.  Find one of them and spend two minutes discussing your problem and possible ways to cure it.

When you complete this step, you may turn to \nbref{second}.
\end{page}


\begin{page}{second}
That was sort of useful.  You don't have a definite answer yet, but you have some idea of how to proceed. Find \iEyebright{}, \iGingko{}, and \iBilberries{}.  Mix them together in the \sCauldron{}.  Brew them for two minutes.

When you complete this step, you may turn to \nbref{third}.
\end{page}


\begin{page}{third}
Curses!!!  The potion stubbornly refuses to turn bright yellow.  The djinni \cWizard{} might know what's wrong.  Show \cWizard{\them} the cauldron and spend two minutes discussing the potion.  If \cWizard{} is unavailable, any other magical creature will do -- you'll just have to spend four minutes discussing instead.

When you complete this step, you may turn to \nbref{fourth}.
\end{page}


\begin{page}{fourth}
Aha, you were missing \iForsythia{}!  Find some and add it to the potion.  Brew for one more minute while making ritual gestures over the cauldron.  (Roleplay accordingly.)

When you complete this step, you may turn to \nbref{fifth}.
\end{page}


\begin{page}{fifth}
Great, the potion is the correct shade of blinding yellow.  Now you need two sorcerous creatures (Fae, Demon, Djinni and/or Lich) to chant together over the potion for one minute.  (Roleplay accordingly.)

When you complete this step, you may turn to \nbref{last}.
\end{page}

\begin{page}{last}
You dip your finger into the potion and wipe it over both eyes.  Then you drink the rest of it.  Bleah, that tasted horrible.  But you can immediately feel the difference.  You are healing! (You have +10 CR for the purpose of \aStoneGaze{}, but for the remainder of game, the duration of stone gaze remains 5 minutes.)
\end{page}

\endnotebook

\end{document}
