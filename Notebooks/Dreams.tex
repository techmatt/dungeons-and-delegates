\documentclass[notebook]{guildcamp2} %% [notebook] or [greennotebook]
\begin{document}

\startnotebook{\nDreams{}}

\begin{page}{first}
Dreams take many forms: sometimes scary, sometimes pleasurable, sometimes thrilling. Monsters and humans have different pleasures and frights, so to better understand the dreams of monsters you will need to find out what they love and fear. Ask two monsters what they most love, and two different monsters what they fear most. Spend at least one minute on each conversation.

When you complete this step, you may turn to \nbref{second}.
\end{page}


\begin{page}{second}
These monsters have such interesting diversity, but you know from experience that, like humans, monsters are never willing to admit what they fear. Use your Read Dreams ability on one of the monsters you interrogated on the previous page, and compare what you observed with what they claimed. Alternatively, you may use Read Dreams on any two monsters.

When you complete this step, you may turn to \nbref{third}.
\end{page}


\begin{page}{third}
Mmm basking in the dreams of others always makes you sleepy. Go find some flowers or herbs and sleep by them for one minute. For the period of time that you are asleep you are unable to resist dark water attacks directed against you.

When you complete this step, you may turn to \nbref{fourth}.
\end{page}


\begin{page}{fourth}
Find something a non-fae monster fears either by asking them, or using your Read Dreams ability (you {\bf cannot} use a monster you have already interrogated from a previous step). Find some way to make their fear a reality. This can be a minor fear, but the recipient must visibly appear to be frightened.

When you complete this step, you may turn to \nbref{fifth}.
\end{page}


\begin{page}{fifth}
Find something a non-fae monster wants either by asking them, or using your Read Dreams ability (you can use a monster you have already interrogated from a previous step). Help them achieve their dream; they must perceive this achievement as a significant goal.

When you complete this step, you may turn to \nbref{last}.
\end{page}

\begin{page}{last}

After reading the dreams of the many monsters here, and realizing both their hopes and fears, you are finally ready to begin manipulating their dreams directly. You may now freely use your Control Dreams ability on Wounded or Knocked Out non-fae monsters (ignore the human-only restriction).

Nothing amuses you more than planting false dreams in humans and watching them walk into your carefully laid traps. You're eager to start experimenting around with monsters.

\end{page}

\endnotebook

\end{document}
