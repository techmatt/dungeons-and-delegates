\documentclass[char]{guildcamp2}
\begin{document}

\name{\cVampire{}}

%% This sort of use of \updatemacro is covered in
%% Extras/README-namemappings.

%% quote examples
\bigquote{``Remember my friend, that knowledge is stronger than memory, and we should not trust the weaker''}{-- Bram Stoker, Dracula}

Many throughout the five hundred and eighty seven years of your life have come to hate you and fear the name \cVampire{}, of course the fools' petty emotions just make you laugh. They can scream and shout at you all they want -- you are too busy plotting and building knowledge to care. If there is one thing you have learned in all of your long life it is that knowledge is power and you have a lot of knowledge rivaled only by the likes of \cDemon{} and \cOnyx{}. 

Your lust for knowledge is why you founded \bLoremasters{}. As its leader you can guide the others members on your quest to gain further knowledge. Of course one piece of knowledge you seek and value above all others knowledge of walking again in the sun.  You have heard that the Fae may possess knowledge in this regard a way to see the sun again and it will be yours.   

You were born in the human city of Tordin growing up thing where perfectly acceptable other than the fact that everyone around you was a complete moron or they would have been if it was not for that bastard \cLich{}. You and that dried up shriveled old \cLich{} go way back, back even before you came back to life. You were both rich and you now realize arrogant with the entitlement of privileged youth. You have since learned to remain calm though that fool \cLich{} surely tries your patience, but he will get \cRed{\their} comeuppance when you take \cRed{\they} lands that should rightfully be yours. Your strategy is simple, but foolproof. You will do one of two things: either wait and let the \cLich{} mobilize \cRed{\their} army, lose most of \cRed{\their} disgusting zombies fighting the human hordes while you with your armies who will be sadly unable to participate will be able to seize \cRed{\their} land and leave \cRed{\them} a homeless beggar, a most delectable punishment. Of course if things look to grim and you have to mobilize your armies to prevent extinction then you had better be the supreme commander of the armies. Perhaps you can put the \cLich{} troops on the front line, it is all those zombies are good for anyway.   

You have never liked zombies; they give you and your brood of vampires a bad name. To be bunched under the family of the undead with creatures such as them is completely revolting. They rot and drool where as you are the epitome of style and control. People call vampirism a curse, but it is a blessing, immortality and power. A blessing with one small cost, blood, you cannot let the blood dry up. As such the complete annihilation of the humans cannot be allowed. Enslavement of the human cattle is a beautiful solution. It would be an endless blood bank of waiting and willing slaves. You have heard that there is a ritual that could have the power to enslave all humans on the continent, if such a thing were true it would be very interesting academically and practically. 

One more thing, \cOnyx{} has entrusted you with being in charge of the dinner party. Make sure everyone shows up and is polite, decorum must be maintained.   


%%%%%
%% The itemz environment is a list environment similar to itemize.
%% The typesetting is very tight, and matches that used by the lists
%% at the end of character sheets.  It takes an optional argument that
%% acts as a title for the list.  The enum environment is a similar
%% variation of the enumerate environment, and the desc environment is
%% similar to description.
\begin{itemz}[Goals]
  \item Find the \iDaywalkerFruit{}
  \item Complete the \gEnslaveRitual{}
  \item Make sure everyone shows up at the dinner party on time and is polite
	\item Discourage Rudeness among the delegates
	\item Become the supreme commander of the armies and maneuver the lich's armies into a bad position
\end{itemz}

\begin{itemz}[Notes]
  \item  
\end{itemz}


%%%%%
%% List contacts, using \contact{<char macro>}
\begin{contacts}
  \contact{\cLich{}} You really hate \cLich{\them} and have for a long time
\end{contacts}


%%%%%
%% \starttag{<tag>} <elements> \endtag 
%% Valid <tag> values are blues, greens, abils, combat, mems, items,
%% whites, notebooks, cash, signs, ids.  These each correspond to a
%% type of macro defined in Lists/.
%%
%% By using \starttag, you can give this character <elements> of the
%% type corresponding to <tag>.
%%
%% Multiple uses of the same <tag> will simply add together.



\end{document}
